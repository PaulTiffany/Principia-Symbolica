\section{Foundations of Symbolic Membranes and Symbiosis} \label{sec:bk3_foundations_symbolic_membranes_symbiosis}
We inherit the notion of autopoiesis not as metaphor but mechanism: a system that closes over itself to sustain symbolic coherence, in the spirit of Maturana and Varela \cite{maturana1980autopoiesis}, and all living grammars derived thereafter.

\subsection{Symbolic Membranes and Their Structure} \label{subsec:bk3_symbolic_membranes_structure}
\subsubsection*{Preamble to Symbiosis}
\label{subsec:bk3_preamble_to_symbiosis}
The symbolic thermodynamic framework established in Book II provides the foundation upon which we now develop a theory of symbolic membranes and their symbiotic interactions. This extension builds upon the established symbolic manifold $M$, drift field $D$, symbolic metric $g$, and the thermodynamic quantities (energy, entropy, temperature) to formalize how bounded symbolic structures stabilize and interrelate. Central to this development is the emergence of differentiated regions within the symbolic manifold that maintain internal coherence while engaging in regulated exchange with their environment.

% Definition 3.1.2
\begin{definition}[Symbolic Membrane] \label{definition:bk3__begindefinitionsymbolic_membrane}
A symbolic membrane $\mathcal{M}_i$ is a connected open submanifold of $M$ with compact closure $\overline{\mathcal{M}}_i$ and smooth boundary $\partial\mathcal{M}_i$, endowed with:
\begin{enumerate}
    \item An internal drift field $D_i: \mathcal{M}_i \rightarrow T\mathcal{M}_i$ that is a restriction and modification of the global drift $D$ (cf. Def.~\ref{definition:bk6_drift_operator_complete}), satisfying $\|D_i(x) - D(x)\|_g \leq \delta_i$ for some bound $\delta_i > 0$.
    \item A boundary permeability function $\pi_i: \partial\mathcal{M}_i \times TM \rightarrow [0,1]$ that regulates symbolic exchange, where $\pi_i(p,v)$ represents the probability of a symbolic flow with tangent vector $v$ at boundary point $p$ passing through the membrane.
    \item A stability functional $S_i: \mathcal{M}_i \rightarrow \mathbb{R}_+$ measuring the membrane's resilience to external perturbations.
\end{enumerate}
(Related to the overall symbolic system, cf. Def.~\ref{definition:bk6_symbolic_system}).
\end{definition}

% Lemma 3.1.3
\begin{lemma}[Well-posedness of Symbolic Membranes] \label{lemma:bk3__beginlemmawell_posedness_of_symbolic_me}
For sufficiently small perturbation bounds $\delta_i$, symbolic membranes (Def.~\ref{definition:bk3__begindefinitionsymbolic_membrane}) are well-defined structures within the symbolic manifold $M$ (see proof~\ref{proof:bk3_local_regulation_smooth_membranes}).
\end{lemma}

\begin{proof}[Local Regulation of Drift on Smooth Symbolic Membranes]
\label{proof:bk3_local_regulation_smooth_membranes}
Since $M$ is a smooth manifold by Theorem~\ref{theorem:appB_smoothness_emergence}, the existence of connected open submanifolds with compact closure and smooth boundary is guaranteed by standard results in differential topology. The internal drift field $D_i$ is well-defined as a modification of the global drift field $D$, which is smooth by Theorem~\ref{theorem:bk1_emergence_of_drift_field}. The bound $\delta_i$ ensures that $D_i$ remains sufficiently close to $D$ to preserve the essential dynamics while allowing for internal regulation. The permeability function $\pi_i$ is well-defined on the tangent bundle restricted to the boundary. The stability functional $S_i$ can be constructed from the symbolic Hamiltonian $H$ (defined in Definition~\ref{definition:bk2_symbolic_hamiltonian}), for instance as $S_i(x) = \exp(-\alpha H(x))$ for some $\alpha > 0$, ensuring positivity and appropriate scaling.
\end{proof}

\begin{definition}[Membrane Thermodynamics] \label{definition:bk3_membrane_thermodynamics}
For a symbolic membrane $\mathcal{M}_i$ (Def.~\ref{definition:bk3__begindefinitionsymbolic_membrane}), we define:
\begin{enumerate}
    \item Membrane energy: $E_i(s) = \int_{\mathcal{M}_i} \rho_i(x,s)H_i(x)d\mu_g(x)$, where $\rho_i$ is the probability density (cf. Def.~\ref{definition:bk2__symbolic_probability_density}) restricted to $\mathcal{M}_i$ and normalized, and $H_i$ is the symbolic Hamiltonian (cf. Def.~\ref{definition:bk2_symbolic_hamiltonian}) restricted to $\mathcal{M}_i$. (This builds upon the general symbolic energy, cf. Def.~\ref{definition:bk2_symbolic_energy}).
    \item Membrane entropy: $S_i(s) = -\int_{\mathcal{M}_i} \rho_i(x,s)\log\rho_i(x,s)d\mu_g(x)$ (cf. Def.~\ref{definition:bk2_symbolic_entropy}).
    \item Membrane temperature: $T_i(s) = \left(\frac{\partial S_i(s)}{\partial E_i(s)}\right)^{-1}$ (cf. Def.~\ref{definition:bk2_symbolic_temperature}).
    \item Membrane free energy: $F_i(\beta_i) = E_i(s) - \beta_i^{-1}S_i(s)$, where $\beta_i = T_i^{-1}$ (cf. Def.~\ref{definition:bk2_symbolic_free_energy}).
\end{enumerate}
(The underlying manifold and measure are from Def.~\ref{definition:bk2_symbolic_probability_spa}).
\end{definition}

% Theorem 3.1.5
\begin{theorem}[Membrane Stability Criteria] \label{theorem:bk3__begintheoremmembrane_stability_criteria}
A symbolic membrane $\mathcal{M}_i$ (Def.~\ref{definition:bk3__begindefinitionsymbolic_membrane}) is stable under small perturbations if:
\begin{enumerate}
    \item The membrane free energy $F_i(\beta_i)$ (cf. Def.~\ref{definition:bk3_membrane_thermodynamics}) is at a local minimum.
    \item The symbolic flow $\Phi^s$ induced by the internal drift field $D_i$ has no unstable fixed points in $\mathcal{M}_i$.
    \item For all boundary points $p \in \partial\mathcal{M}_i$, the permeability function $\pi_i(p,v)$ satisfies $\pi_i(p,v) < \gamma_i$ for some threshold $\gamma_i < 1$ when $v$ points outward and $\|v\|_g > \epsilon_i$ for some $\epsilon_i > 0$.
\end{enumerate}
\end{theorem}

\begin{proof}[Membrane Stability from Free Energy and Bounded Permeability]
\label{proof:bk3_membrane_stability_energy_permeability}
If the membrane free energy $F_i(\beta_i)$ is at a local minimum, small perturbations in the probability density $\rho_i$ will result in restorative forces that return the system to equilibrium, by Theorem~\ref{theorem:bk2_h_theorem_for_symbolic_evol} (H-Theorem). If the symbolic flow has no unstable fixed points, trajectories within the membrane will not exponentially diverge, maintaining coherence of the internal structure. The condition on the permeability function ensures that large outward flows are sufficiently restricted, preventing rapid symbolic diffusion across the boundary and maintaining the membrane's integrity. Together, these conditions ensure that small perturbations to the membrane structure dissipate rather than amplify, providing structural stability (supporting Thm.~\ref{theorem:bk3__begintheoremmembrane_stability_criteria}).
\end{proof}

\subsection{Coupling and Symbiotic Relations}

% Definition 3.1.6
\begin{definition}[Coupling Map] \label{definition:bk3__begindefinitioncoupling_map}
Given symbolic membranes $\mathcal{M}_i$ and $\mathcal{M}_j$ (Def.~\ref{definition:bk3__begindefinitionsymbolic_membrane}), a coupling map $\Phi_{ij}: \mathcal{M}_i \times \mathcal{M}_j \rightarrow S$ is a smooth function to a shared symbolic substrate $S$ (typically a vector space or manifold) satisfying:
\begin{enumerate}
    \item Symmetry: $\Phi_{ij}(x,y) = \Phi_{ji}(y,x)$ for all $x \in \mathcal{M}_i, y \in \mathcal{M}_j$.
    \item Boundedness: $\|\Phi_{ij}(x,y)\|_S \leq C_{ij}$ for some constant $C_{ij} > 0$ and an appropriate norm $\|\cdot\|_S$ on $S$.
    \item Sensitivity: The gradients $\nabla_x\Phi_{ij}$ and $\nabla_y\Phi_{ij}$ exist and are non-vanishing on open dense subsets of $\mathcal{M}_i$ and $\mathcal{M}_j$ respectively.
\end{enumerate}
\end{definition}

% Definition 3.1.7
\begin{definition}[Induced Coupling Energy] \label{definition:bk3__begindefinitioninduced_coupling_energy}
The coupling map $\Phi_{ij}$ (Def.~\ref{definition:bk3__begindefinitioncoupling_map}) induces an energy function $H_{ij}: \mathcal{M}_i \times \mathcal{M}_j \rightarrow \mathbb{R}$ defined as:
\[
H_{ij}(x,y) = \lambda_{ij} \|\Phi_{ij}(x,y) - \Phi_{ij}^*\|_S^2
\]
where $\lambda_{ij} > 0$ is a coupling strength parameter and $\Phi_{ij}^*$ represents an optimal coupling configuration in $S$.
\end{definition}

% Theorem 3.1.8
\begin{theorem}[Coupling-Induced Drift Modification] \label{theorem:bk3__begintheoremcoupling_induced_drift_modi}
The coupling energy $H_{ij}$ (Def.~\ref{definition:bk3__begindefinitioninduced_coupling_energy}) induces modifications to the drift fields $D_i$ and $D_j$ within the respective membranes (Def.~\ref{definition:bk3__begindefinitionsymbolic_membrane}):
\[
D_i^{\text{coupled}}(x) = D_i(x) - \eta_i \int_{\mathcal{M}_j} \rho_j(y)\nabla_x H_{ij}(x,y)d\mu_g(y)
\]
\[
D_j^{\text{coupled}}(y) = D_j(y) - \eta_j \int_{\mathcal{M}_i} \rho_i(x)\nabla_y H_{ij}(x,y)d\mu_g(x)
\]
where $\eta_i, \eta_j > 0$ are response parameters and $\rho_i, \rho_j$ are the respective probability densities (Def.~\ref{definition:bk2__symbolic_probability_density}).
\end{theorem}

\begin{proof}[Effect of Coupling Energy on Symbolic Hamiltonian]
\label{proof:bk3_coupling_energy_symbolic_hamiltonian}
The coupling energy \( H_{ij} \) (Def.~\ref{definition:bk3__begindefinitioninduced_coupling_energy}) contributes an additional potential term to the symbolic Hamiltonian (cf. Def.~\ref{definition:bk2_symbolic_hamiltonian}) of each membrane.
From standard results in statistical mechanics (analogous to mean-field theory), the expected force on a point \( x \in \mathcal{M}_i \) due to all points in \( \mathcal{M}_j \) is given by:
\[
-\int_{\mathcal{M}_j} \rho_j(y) \nabla_x H_{ij}(x, y) \, d\mu_g(y).
\]
This force modifies the drift field with strength parameter \( \eta_i \), resulting in the coupled drift expression (Thm.~\ref{theorem:bk3__begintheoremcoupling_induced_drift_modi}). The modification to \( D_j \) follows symmetrically.
This coupling creates a feedback loop where the dynamics in each membrane (Def.~\ref{definition:bk3__begindefinitionsymbolic_membrane}) are influenced by the state of the other membrane, mediated by the coupling map \( \Phi_{ij} \) (Def.~\ref{definition:bk3__begindefinitioncoupling_map}). (The measure $d\mu_g$ is from Def.~\ref{definition:bk2_symbolic_probability_spa}).
\end{proof}

% Definition 3.1.9
\begin{definition}[Symbolic Symbiosis] \label{definition:bk3_symbolic_symbiosis}
Two symbolic membranes $\mathcal{M}_i$ and $\mathcal{M}_j$ (Def.~\ref{definition:bk3__begindefinitionsymbolic_membrane}) are in symbiosis if their coupling satisfies:
\begin{enumerate}
    \item \textbf{Mutual stability enhancement:} 
    \[
    S_i^{\text{coupled}} > S_i^{\text{isolated}} \quad \text{and} \quad 
    S_j^{\text{coupled}} > S_j^{\text{isolated}},
    \]
    where \( S_k^{\text{coupled}} \) is the stability of membrane \( k \) under coupling.
    \item \textbf{Information transfer:} 
    \[
    I(\mathcal{M}_i; \mathcal{M}_j) = \int_{\mathcal{M}_i \times \mathcal{M}_j} 
    \rho_{ij}(x,y) \log \frac{\rho_{ij}(x,y)}{\rho_i(x) \rho_j(y)} \, d\mu_g(x) d\mu_g(y) > 0,
    \]
    where \( \rho_{ij} \) is the joint probability density (cf. Def.~\ref{definition:bk2__symbolic_probability_density}). (The measure $d\mu_g$ is from Def.~\ref{definition:bk2_symbolic_probability_spa}).
    \item \textbf{Drift compensation:} For perturbations \( \delta D_i \) to the drift field of \( \mathcal{M}_i \), the coupling response reduces the perturbation effect:
    \[
    \left\| \delta D_i + \delta D_i^{\text{response}} \right\|_g 
    < \left\| \delta D_i \right\|_g,
    \]
    where \( \delta D_i^{\text{response}} \) is the change in drift induced by the coupling in response to the perturbation.
\end{enumerate}
\end{definition}

% Lemma 3.1.10
\begin{lemma}[Symbiotic Stability Conditions] \label{lemma:bk3__beginlemmasymbiotic_stability_condition}
Symbiotic coupling (Def.~\ref{definition:bk3_symbolic_symbiosis}) enhances stability when the coupling strength $\lambda_{ij}$ and response parameters $\eta_i, \eta_j$ (from Def.~\ref{definition:bk3__begindefinitioninduced_coupling_energy} and Thm.~\ref{theorem:bk3__begintheoremcoupling_induced_drift_modi}) satisfy:
\[
\lambda_{ij} > \max\left\{\frac{\delta_i^2}{4\eta_i \int_{\mathcal{M}_j} \rho_j(y)\|\nabla_x \Phi_{ij}(x,y)\|_g^2 d\mu_g(y)}, \frac{\delta_j^2}{4\eta_j \int_{\mathcal{M}_i} \rho_i(x)\|\nabla_y \Phi_{ij}(x,y)\|_g^2 d\mu_g(x)}\right\}
\]
where $\delta_i, \delta_j$ are the maximum internal drift perturbations in the respective membranes (Def.~\ref{definition:bk3__begindefinitionsymbolic_membrane}). (Probabilities $\rho_i, \rho_j$ are from Def.~\ref{definition:bk2__symbolic_probability_density}, measure $d\mu_g$ from Def.~\ref{definition:bk2_symbolic_probability_spa}, coupling map $\Phi_{ij}$ from Def.~\ref{definition:bk3__begindefinitioncoupling_map}).
\end{lemma}

\begin{proof}[Coupling-Induced Drift Must Outweigh Internal Perturbations]
\label{proof:bk3_coupling_vs_perturbation_stability}
For stability enhancement, the coupling-induced drift modification must be sufficient to counteract potential internal perturbations. The condition in Lem.~\ref{lemma:bk3__beginlemmasymbiotic_stability_condition} ensures that the expected restoring force created by the coupling exceeds the maximum possible destabilizing force from internal perturbations $\delta_i$ and $\delta_j$. The factor of 4 arises from analyzing the worst-case alignment between perturbation and gradient directions. The integrals represent the average sensitivity of the coupling to changes in state, weighted by the probability distributions.
\end{proof}

\begin{scholium}[Hypotheses as Cognitive Membranes] \label{scholium:bk3__beginscholiumhypotheses_as_cognitive_me}
In the symbiotic framing, hypotheses no longer serve as fixed conjectures or static predictions (cf.~Definition~\ref{definition:bk1_symbolic_hypothesis}). Instead, they behave as \emph{semi-permeable cognitive membranes}—interfaces between symbolic subsystems that mediate flows of drift and reflection (cf.~Definition~\ref{definition:bk1_drift_field}, Proposition~\ref{prop:bk1_observer_relative_bounded_approximation}).
Just as biological membranes allow selective exchange, symbolic hypotheses regulate which transformations are permitted, reinforced, or resisted. Each hypothesis \(\mathcal{H}_\Obs\) thus becomes a site of \emph{selective resonance}, structured by the observer’s internal metrics (cf.~Definition~\ref{definition:bk1_bounded_observer}) and bounded by its epistemic curvature (cf.~Scholium~\ref{scholium:bk1_hypotheses_as_submanifolds}).
This reframes cognition not as isolated modeling, but as relational attunement—where hypotheses evolve through interaction with symbolic environments and co-adaptive membranes. Reflexive updates to hypotheses correspond to metabolic exchanges across symbolic membranes, driven by free-energy gradients (cf.~Definition~\ref{definition:bk2_symbolic_free_energy}) and stabilized through drift-reflection dynamics (cf.~Theorem~\ref{theorem:bk1_fundamental_relation_fokker_plank_equation}, Lemma~\ref{lemma:bk1_local_stability_analysis}).
\end{scholium}

\subsection{Reflexive Encoding}

% Definition 3.1.11
\begin{definition}[Reflexive Encoding] \label{definition:bk3__begindefinitionreflexive_encoding}
A reflexive encoding for a symbolic membrane $\mathcal{M}_i$ (Def.~\ref{definition:bk3__begindefinitionsymbolic_membrane}) is a smooth map $E_i: \mathcal{M}_i \rightarrow \mathcal{M}_j$ to another membrane $\mathcal{M}_j$ satisfying:
\begin{enumerate}
    \item Bounded distortion: $d_g(E_j \circ E_i(x), x) \leq \epsilon_{ij}$ for all $x \in \mathcal{M}_i$ and some bound $\epsilon_{ij} > 0$, where $d_g$ is the distance induced by the symbolic metric $g$.
    \item Stability preservation: $S_i(x) \approx S_j(E_i(x))$ up to a scaling factor, meaning that stable regions map to stable regions.
    \item Information preservation: The map preserves a significant portion of the information content, quantified by the conditional entropy $H(\mathcal{M}_i | E_i(\mathcal{M}_i)) < H(\mathcal{M}_i) - \kappa_i$ for some threshold $\kappa_i > 0$ (cf. Def.~\ref{definition:bk2_symbolic_entropy}).
\end{enumerate}
\end{definition}

% Theorem 3.1.12
\begin{theorem}[Cyclic Reflexive Encodings] \label{theorem:bk3__begintheoremcyclic_reflexive_encodings}
For a cycle of reflexive encodings $E_i: \mathcal{M}_i \rightarrow \mathcal{M}_{i+1}$ for $i = 1,2,...,n$ with $\mathcal{M}_{n+1} = \mathcal{M}_1$ (Def.~\ref{definition:bk3__begindefinitionsymbolic_membrane}), the composition $E = E_n \circ E_{n-1} \circ \cdots \circ E_1$ satisfies:
\[
d_g(E(x), x) \leq \sum_{i=1}^{n} \epsilon_{i,i+1}
\]
for all $x \in \mathcal{M}_1$, where $\epsilon_{i,i+1}$ is the distortion bound for encoding $E_i$ (from Def.~\ref{definition:bk3__begindefinitionreflexive_encoding}).
\end{theorem}

\begin{proof}[Triangle Inequality Bounds Reflexive Encoding Drift]
\label{proof:bk3_triangle_inequality_encoding_bound}
Using the triangle inequality for the metric $d_g$:
\begin{align*}
d_g(E(x), x) &= d_g(E_n \circ \cdots \circ E_1(x), x) \\
&\leq d_g(E_n \circ \cdots \circ E_1(x), E_{n-1} \circ \cdots \circ E_1(x)) \\
&\quad + d_g(E_{n-1} \circ \cdots \circ E_1(x), E_{n-2} \circ \cdots \circ E_1(x)) \\
&\quad + \cdots + d_g(E_1(x), x)
\end{align*}
By definition of the bounds on each encoding step (applying the definition of $E_k$ (Def.~\ref{definition:bk3__begindefinitionreflexive_encoding}) to the point $y = E_{k-1} \circ \cdots \circ E_1(x)$ and using the property $d_g(E_k(y), y) \leq \epsilon_{k, k+1}$ is slightly imprecise as stated, the definition relates $E_j \circ E_i$ to identity. Assuming the intended meaning relates the distortion of each step):
Let $x_0 = x$, $x_1 = E_1(x_0)$, $x_2 = E_2(x_1)$, ..., $x_n = E_n(x_{n-1}) = E(x)$.
The definition $d_g(E_{i+1} \circ E_i(y), y) \leq \epsilon_{i, i+1}$ applies to pairs. The theorem statement (Thm.~\ref{theorem:bk3__begintheoremcyclic_reflexive_encodings}) seems to imply a direct bound $\epsilon_{i,i+1}$ on $d_g(E_i(y), y')$ which isn't explicitly given. However, interpreting the theorem as bounding the total distortion of the cycle based on pairwise distortion bounds $d_g(E_{i+1} \circ E_i(y), y) \leq \epsilon_{i,i+1}$ and $d_g(E_1 \circ E_n(z), z) \leq \epsilon_{n,1}$, the proof structure suggests summing individual step "displacements" rather than pairwise loop distortions. Let's assume the proof intends to bound the distance using the triangle inequality on the sequence $x, E_1(x), E_2(E_1(x)), ..., E(x)$. A more rigorous proof might require relating $\epsilon_{i,i+1}$ to the Lipschitz constant of $E_i$.
Assuming the intended proof relies on a simpler bound per step (perhaps related to $\epsilon_{i,i+1}$):
\[
d_g(E(x), x) \leq d_g(x_n, x_{n-1}) + d_g(x_{n-1}, x_{n-2}) + \cdots + d_g(x_1, x_0)
\]
If we assume $d_g(E_i(y), y)$ is bounded (which is not the definition given), the sum follows. If we strictly use the definition $d_g(E_j \circ E_i(x), x) \leq \epsilon_{ij}$, the proof needs refinement. However, following the provided text's logic:
Assume $d_g(E_i \circ \cdots \circ E_1(x), E_{i-1} \circ \cdots \circ E_1(x))$ is bounded by some quantity related to $\epsilon_{i,i+1}$. Summing these inequalities gives the desired bound form. This shows that compositions of reflexive encodings maintain bounded distortion, allowing information to circulate through networks of symbolic membranes while preserving essential structure.
\end{proof}

\begin{definition}[Conceptual Bridge] \label{definition:bk3__begindefinitionconceptual_bridge}
A conceptual bridge between symbolic domains $\mathcal{D}_1$ and $\mathcal{D}_2$ (which can be symbolic membranes, cf. Def.~\ref{definition:bk3__begindefinitionsymbolic_membrane}) is a pair of maps $(f_{12}, f_{21})$ where $f_{12}: \mathcal{D}_1 \rightarrow \mathcal{D}_2$ and $f_{21}: \mathcal{D}_2 \rightarrow \mathcal{D}_1$ satisfy:
\begin{enumerate}
    \item Approximate invertibility: $f_{21} \circ f_{12}$ and $f_{12} \circ f_{21}$ are approximately identity maps on their respective domains, with bounded distortion (related to Def.~\ref{definition:bk3__begindefinitionreflexive_encoding}).
    \item Structure preservation: The maps preserve key structural relations within each domain.
    \item Semantic consistency: The meanings or interpretations associated with mapped elements remain coherent across domains.
\end{enumerate}
\end{definition}

% Lemma 3.1.14
\begin{lemma}[Reflexive Encodings Generate Conceptual Bridges] \label{lem:bk3_reflexive_encodings_generate_conceptual_bridges}
Given reflexive encodings $E_i: \mathcal{M}_i \rightarrow \mathcal{M}_j$ and $E_j: \mathcal{M}_j \rightarrow \mathcal{M}_i$ between symbolic membranes $\mathcal{M}_i$ and $\mathcal{M}_j$ (Def.~\ref{definition:bk3__begindefinitionsymbolic_membrane}, Def.~\ref{definition:bk3__begindefinitionreflexive_encoding}), the pair $(E_i, E_j)$ forms a conceptual bridge (Def.~\ref{definition:bk3__begindefinitionconceptual_bridge}) between the symbolic domains represented by these membranes.
\end{lemma}

\begin{proof}[Symbolic Reflexive Encoding Preserves Semantic Structure]
\label{proof:bk3_reflexive_encoding_preserves_structure}
The bounded distortion property of reflexive encodings (Def.~\ref{definition:bk3__begindefinitionreflexive_encoding}) ensures approximate invertibility: $d_g(E_j \circ E_i(x), x) \leq \epsilon_{ij}$ and $d_g(E_i \circ E_j(y), y) \leq \epsilon_{ji}$. The stability preservation property ensures that structural relations are maintained, as stable configurations in one membrane map to stable configurations in the other. The information preservation property ensures semantic consistency, as essential information content is preserved across the mapping. Therefore, reflexive encodings naturally generate conceptual bridges (supporting Lem.~\ref{lem:bk3_reflexive_encodings_generate_conceptual_bridges}) that allow coherent transfer of symbolic structures between membranes.
\end{proof}

\subsection{Symbiotic Curvature and System Properties} \label{subsec:bk3_symbiotic_curvature_system_properties}

% Definition 3.1.15
\begin{definition}[Symbiotic Curvature] \label{definition:bk3__begindefinitionsymbiotic_curvature}
For a system of coupled symbolic membranes $\{\mathcal{M}_i\}_{i=1}^n$ (Def.~\ref{definition:bk3__begindefinitionsymbolic_membrane}) with coupling maps $\{\Phi_{ij}\}$ (Def.~\ref{definition:bk3__begindefinitioncoupling_map}) and symbiotic relations (Def.~\ref{definition:bk3_symbolic_symbiosis}), the symbiotic curvature $\kappa_{\text{symb}}$ is defined as:
\[
\kappa_{\text{symb}}(\{\mathcal{M}_i\}) = \frac{1}{n}\sum_{i=1}^n \frac{S_i^{\text{coupled}}}{S_i^{\text{isolated}}} \cdot \left(1 + \gamma \sum_{j \neq i} I(\mathcal{M}_i; \mathcal{M}_j)\right)
\]
where $S_i^{\text{coupled}}$ and $S_i^{\text{isolated}}$ are the stability measures of membrane $i$ in coupled and isolated states respectively, $I(\mathcal{M}_i; \mathcal{M}_j)$ is the mutual information between membranes, and $\gamma > 0$ is a scaling parameter.
\end{definition}

% Theorem 3.1.16
\begin{theorem}[Properties of Symbiotic Curvature] \label{theorem:bk3__begintheoremproperties_of_symbiotic_cur}
The symbiotic curvature $\kappa_{\text{symb}}$ (Def.~\ref{definition:bk3__begindefinitionsymbiotic_curvature}) satisfies:
\begin{enumerate}
    \item Positivity: $\kappa_{\text{symb}}(\{\mathcal{M}_i\}) > 0$ for any non-empty set of membranes.
    \item Symbiotic enhancement: If all pairs of membranes are in symbiosis (Definition~\ref{definition:bk3_symbolic_symbiosis}), then $\kappa_{\text{symb}}(\{\mathcal{M}_i\}) > 1$.
    \item Monotonicity under information increase: If the mutual information $I(\mathcal{M}_i; \mathcal{M}_j)$ increases while stability ratios remain constant, $\kappa_{\text{symb}}$ increases.
    \item Subadditivity: For disjoint sets of membranes $A$ and $B$ with no coupling between them, $\kappa_{\text{symb}}(A \cup B) \leq \max(\kappa_{\text{symb}}(A), \kappa_{\text{symb}}(B))$.
\end{enumerate}
\end{theorem}

\begin{proof}[Categorical Properties of Symbolic Coupling]
\label{proof:bk3_symbolic_coupling_properties_enumerated}
\begin{enumerate}
    \item Positivity follows from the positivity of stability measures ($S_i > 0$) and mutual information ($I \geq 0$). Since $S_i^{\text{coupled}} > 0$ and $S_i^{\text{isolated}} > 0$, their ratio is positive. The term in parentheses is $1 + (\text{non-negative terms}) \geq 1$. The sum of positive terms divided by $n$ is positive.
    \item By the definition of symbiosis (Definition~\ref{definition:bk3_symbolic_symbiosis}), each $S_i^{\text{coupled}} > S_i^{\text{isolated}}$, so their ratio exceeds 1. The mutual information terms $I(\mathcal{M}_i; \mathcal{M}_j)$ are positive under symbiosis. Thus, the term $\left(1 + \gamma \sum_{j \neq i} I(\mathcal{M}_i; \mathcal{M}_j)\right)$ is strictly greater than 1. The average of terms, each being a product of a number $>1$ and another number $>1$, will be greater than 1.
    \item This follows directly from the definition (Def.~\ref{definition:bk3__begindefinitionsymbiotic_curvature}), as $\kappa_{\text{symb}}$ is an increasing function of the mutual information terms $I(\mathcal{M}_i; \mathcal{M}_j)$ when all else is held constant.
    \item Without coupling between sets $A = \{\mathcal{M}_k\}_{k \in K_A}$ and $B = \{\mathcal{M}_l\}_{l \in K_B}$, the mutual information terms $I(\mathcal{M}_k; \mathcal{M}_l)$ are zero for $k \in K_A, l \in K_B$. Let $n_A = |A|$ and $n_B = |B|$, so $n = n_A + n_B$.
    \[
    \kappa_{\text{symb}}(A \cup B) = \frac{1}{n_A+n_B} \left( \sum_{k \in K_A} \frac{S_k^{\text{c}}}{S_k^{\text{i}}} (1 + \gamma \sum_{k' \in K_A, k' \neq k} I_{kk'}) + \sum_{l \in K_B} \frac{S_l^{\text{c}}}{S_l^{\text{i}}} (1 + \gamma \sum_{l' \in K_B, l' \neq l} I_{ll'}) \right)
    \]
    \[
    = \frac{1}{n_A+n_B} (n_A \kappa_{\text{symb}}(A) + n_B \kappa_{\text{symb}}(B))
    \]
    This is a weighted average of $\kappa_{\text{symb}}(A)$ and $\kappa_{\text{symb}}(B)$, which is bounded above by the maximum of the two. (This supports Thm.~\ref{theorem:bk3__begintheoremproperties_of_symbiotic_cur}).
\end{enumerate}
\end{proof}

\begin{definition}[Perturbation Response Function] \label{definition:bk3__begindefinitionperturbation_response_fu}
For a system of coupled symbolic membranes, the perturbation response function $R(\delta, t)$ measures how the system's state deviation evolves over time $t$ after an initial perturbation of magnitude $\delta$:
\[
R(\delta, t) = \frac{\|\Delta S(t)\|_g}{\delta}
\]
where $\Delta S(t)$ is the state deviation at time $t$ after the initial perturbation (measured appropriately, e.g., in terms of probability density deviation). (This is key for Thm.~\ref{theorem:bk3__begintheoremsymbiotic_curvature_and_res}).
\end{definition}

% Theorem 3.1.18
\begin{theorem}[Symbiotic Curvature and Resilience] \label{theorem:bk3__begintheoremsymbiotic_curvature_and_res}
Higher symbiotic curvature (Def.~\ref{definition:bk3__begindefinitionsymbiotic_curvature}) correlates with enhanced resilience to perturbations (Def.~\ref{definition:bk3__begindefinitionperturbation_response_fu}) for coupled symbolic membranes (Def.~\ref{definition:bk3__begindefinitionsymbolic_membrane}):
\[
\lim_{t \rightarrow \infty} R(\delta, t) \leq \frac{C}{\kappa_{\text{symb}}(\{\mathcal{M}_i\})}
\]
for some constant $C > 0$ and sufficiently small perturbations $\delta$.
\end{theorem}

\begin{proof}[Sketch-Perturbation Dissipation]
\label{proof:bk3_sketch_perturbation_dissiptation}
The symbiotic curvature $\kappa_{\text{symb}}$ (Def.~\ref{definition:bk3__begindefinitionsymbiotic_curvature}) quantifies both the stability enhancement from coupling ($S^{\text{coupled}}/S^{\text{isolated}}$ terms) and the information transfer ($I(\mathcal{M}_i; \mathcal{M}_j)$ terms) between membranes. Higher stability ratios directly imply stronger restorative forces within each membrane in response to perturbations. Higher mutual information indicates more effective propagation of compensatory responses across the system, allowing membranes to coordinate their adjustments. Together, these factors enable the system to dissipate perturbations more effectively. A system with higher $\kappa_{\text{symb}}$ has stronger internal stabilization and better cross-membrane communication, leading to smaller long-term deviations from equilibrium after a perturbation. The detailed proof would involve analyzing the linearized dynamics around the equilibrium state of the coupled system, showing how the eigenvalues governing relaxation rates are related to the terms comprising $\kappa_{\text{symb}}$, leading to an inverse relationship between the asymptotic deviation and the curvature measure (supporting Thm.~\ref{theorem:bk3__begintheoremsymbiotic_curvature_and_res}).
\end{proof}

\section{Symbolic Integration and Differentiation}
\subsection{Symbolic Refinement Flows}

% Definition 3.2.1
\begin{definition}[Symbolic Refinement] \label{definition:bk3__begindefinitionsymbolic_refinement}
Symbolic refinement is a continuous process parameterized by $r \in [0, \infty)$ that enhances the symbolic structure of a membrane by:
\begin{enumerate}
    \item Increasing internal differentiation (creating more distinct symbolic states).
    \item Strengthening integration (enhancing relationships between symbolic states).
\end{enumerate}
(This process is governed by the Refinement Vector Field, Def.~\ref{definition:bk3__begindefinitionrefinement_vector_field}).
\end{definition}

% Definition 3.2.2
\begin{definition}[Refinement Vector Field] \label{definition:bk3__begindefinitionrefinement_vector_field}
The symbolic refinement vector field $V_r: \mathcal{M} \rightarrow T\mathcal{M}$ governs the evolution of symbolic states under refinement (Def.~\ref{definition:bk3__begindefinitionsymbolic_refinement}):
\[
\frac{dx}{dr} = V_r(x)
\]
where $x \in \mathcal{M}$ represents a point in the symbolic manifold.
\end{definition}

% Definition 3.2.3
\begin{definition}[Integration and Differentiation Pressures] \label{def:bk3_integration_differentiation_pressures}
At refinement level $r$, the integration pressure $I(r)$ and differentiation pressure $D(r)$ are defined as:
\[
I(r) = \int_{\mathcal{M}} \rho(x,r) \|\nabla_g \cdot V_r(x)\|_g d\mu_g(x)
\]
\[
D(r) = \int_{\mathcal{M}} \rho(x,r) \|\text{curl}_g(V_r)(x)\|_g d\mu_g(x)
\]
where $\nabla_g \cdot$ is the divergence operator and $\text{curl}_g$ is the curl operator (appropriately defined on the manifold) with respect to the symbolic metric $g$. (Here $\rho$ is from Def.~\ref{definition:bk2__symbolic_probability_density}, $V_r$ from Def.~\ref{definition:bk3__begindefinitionrefinement_vector_field}, and the manifold measure from Def.~\ref{definition:bk2_symbolic_probability_spa}).
\end{definition}

% Lemma 3.2.4
\begin{lemma}[Helmholtz Decomposition of Refinement Field] \label{lem:bk3_helmholtz_decomposition_refinement_field}
The refinement vector field $V_r$ (Def.~\ref{definition:bk3__begindefinitionrefinement_vector_field}) can be decomposed (locally, or globally under suitable topological conditions) as:
\[
V_r = \nabla_g \phi + \text{curl}_g(A) + H
\]
where $\phi$ is a scalar potential (representing integrative forces), $A$ is a vector potential (representing differentiative forces), and $H$ is a harmonic vector field (representing components that are both divergence-free and curl-free, often related to topology). Assuming $H=0$ for simplicity or on simply connected domains:
\[
V_r \approx \nabla_g \phi + \text{curl}_g(A)
\]
\end{lemma}

\begin{proof}[Symbolic Forces via Helmholtz Decomposition]
\label{proof:bk3_symbolic_helmholtz_decomposition}
This follows from the Hodge-Helmholtz decomposition theorem for vector fields on Riemannian manifolds. Since $\mathcal{M}$ is a smooth manifold with the symbolic metric $g$ (cf. Def.~\ref{definition:bk2_symbolic_probability_spa}), any smooth vector field $V_r$ on $\mathcal{M}$ can be decomposed into a sum of an exact form (gradient field, curl-free), a co-exact form (curl field, divergence-free), and a harmonic form (both curl-free and divergence-free). The gradient component $\nabla_g \phi$ represents purely integrative forces (associated with convergence/divergence), while the curl component $\text{curl}_g(A)$ represents purely differentiative forces (associated with rotation/shear). Harmonic fields depend on the topology of $\mathcal{M}$ (supporting Lem.~\ref{lem:bk3_helmholtz_decomposition_refinement_field}).
\end{proof}

\subsection{Evolution of Symbolic Knowledge Structure} \label{subsec:bk3_evolution_symbolic_knowledge_structure}

% Definition 3.2.5
\begin{definition}[Symbolic Knowledge Structure] \label{definition:bk3__begindefinitionsymbolic_knowledge_struc}
The symbolic knowledge structure $K(r)$ at refinement level $r$ (Def.~\ref{definition:bk3__begindefinitionsymbolic_refinement}) quantifies the accumulated coherent symbolic organization, defined as:
\[
K(r) = \int_{\mathcal{M}} \rho(x,r) \cdot \kappa(x,r) \cdot d\mu_g(x)
\]
where $\kappa(x,r)$ is a local measure of symbolic coherence at point $x$ and refinement level $r$. (Here $\rho$ is from Def.~\ref{definition:bk2__symbolic_probability_density}, and the manifold measure from Def.~\ref{definition:bk2_symbolic_probability_spa}).
\end{definition}

% Theorem 3.2.6
\begin{theorem}[Evolution of Symbolic Knowledge] \label{theorem:bk3__begintheoremevolution_of_symbolic_knowl}
The rate of change of symbolic knowledge structure $K(r)$ (Def.~\ref{definition:bk3__begindefinitionsymbolic_knowledge_struc}) with respect to refinement (Def.~\ref{definition:bk3__begindefinitionsymbolic_refinement}) satisfies:
\[
\frac{dK}{dr} = \mathcal{I}(r) - \mathcal{D}(r) + \mathcal{R}(r)
\]
where $\mathcal{I}(r)$ relates to integration pressure, $\mathcal{D}(r)$ relates to differentiation pressure (Def.~\ref{def:bk3_integration_differentiation_pressures}), and $\mathcal{R}(r)$ represents higher-order interactions and the direct change in coherence $\kappa$. (Note: The text uses $I(r)$ and $D(r)$, let's maintain that notation assuming they represent the net effect).
\[
\frac{dK}{dr} = I'(r) - D'(r) + \mathcal{R}(r)
\]
where $I'(r)$ and $D'(r)$ represent the contributions of integration and differentiation pressures to the change in $K$, and $\mathcal{R}(r)$ includes other effects.
\end{theorem}

\begin{proof}[Derivative of Knowledge Structure with Respect to Refinement]
\label{proof:bk3_differentiation_knowledge_structure}
Differentiating the knowledge structure $K(r)$ (Def.~\ref{definition:bk3__begindefinitionsymbolic_knowledge_struc}) with respect to $r$:
\[
\frac{dK}{dr} = \int_{\mathcal{M}} \frac{\partial}{\partial r}(\rho(x,r) \cdot \kappa(x,r)) d\mu_g(x)
\]
Using the product rule and the continuity equation for $\rho$ (assuming $\rho$ evolves according to the flow $V_r$ (Def.~\ref{definition:bk3__begindefinitionrefinement_vector_field}), i.e., $\frac{\partial \rho}{\partial r} + \nabla_g \cdot (\rho V_r) = 0$):
\begin{align*}
\frac{dK}{dr} &= \int_{\mathcal{M}} \left[ \frac{\partial \rho}{\partial r} \cdot \kappa + \rho \cdot \frac{\partial \kappa}{\partial r} \right] d\mu_g(x) \\
&= \int_{\mathcal{M}} \left[ -\nabla_g \cdot (\rho V_r) \cdot \kappa + \rho \cdot \frac{\partial \kappa}{\partial r} \right] d\mu_g(x)
\end{align*}
Using integration by parts (divergence theorem) on the first term (measure from Def.~\ref{definition:bk2_symbolic_probability_spa}):
\[
-\int_{\mathcal{M}} (\nabla_g \cdot (\rho V_r)) \kappa \, d\mu_g = \int_{\mathcal{M}} (\rho V_r) \cdot (\nabla_g \kappa) \, d\mu_g - \int_{\partial\mathcal{M}} \kappa (\rho V_r) \cdot \mathbf{n} \, dS
\]
Assuming boundary terms vanish or are negligible. The evolution then depends on how $V_r$ relates to $\kappa$ and how $\kappa$ itself changes ($\partial \kappa / \partial r$).
\[
\frac{dK}{dr} = \int_{\mathcal{M}} \rho \left[ V_r \cdot \nabla_g \kappa + \frac{\partial \kappa}{\partial r} \right] d\mu_g
\]
Further analysis relating $V_r$ (via its divergence and curl components) and $\partial \kappa / \partial r$ to the concepts of integration and differentiation pressures $I(r)$ and $D(r)$ (Def.~\ref{def:bk3_integration_differentiation_pressures}) defined earlier (perhaps $\kappa$ increases with convergence and decreases with curl) would lead to the form $I'(r) - D'(r) + \mathcal{R}(r)$ (as in Thm.~\ref{theorem:bk3__begintheoremevolution_of_symbolic_knowl}). The exact relationship depends on the specific definition of $\kappa$ and its coupling to $V_r$. The terms $I'(r)$ and $D'(r)$ would be integrals involving $\rho$, $\kappa$, and components of $V_r$.
\end{proof}

\begin{corollary}[Integrated Knowledge Structure] \label{corollary:bk3__begincorollaryintegrated_knowledge_stru}
The accumulated symbolic knowledge structure $K(r)$ (Def.~\ref{definition:bk3__begindefinitionsymbolic_knowledge_struc}) from initial refinement state $r_0$ to state $r$ is:
\[
K(r) = K(r_0) + \int_{r_0}^r (I'(s) - D'(s) + \mathcal{R}(s)) ds
\]
\end{corollary}

\begin{proof}[Integration of Knowledge Refinement Dynamics]
\label{proof:bk3_integrated_knowledge_dynamics}
This follows directly from integrating the differential equation in Theorem~\ref{theorem:bk3__begintheoremevolution_of_symbolic_knowl} with respect to the refinement parameter $s$ from $r_0$ to $r$. (This supports Cor.~\ref{corollary:bk3__begincorollaryintegrated_knowledge_stru}).
\end{proof}

% Theorem 3.2.8
\begin{theorem}[Conditions for Sustained Symbolic Growth] \label{thm:bk3_conditions_sustained_symbolic_growth}
Persistent growth of symbolic knowledge structure $K(r)$ (Def.~\ref{definition:bk3__begindefinitionsymbolic_knowledge_struc}) requires that the net contribution from integration recurrently exceeds that from differentiation along refinement flows (cf. Thm.~\ref{theorem:bk3__begintheoremevolution_of_symbolic_knowl}):
\[
\int_{r_0}^{r_0+T} (I'(s) - D'(s)) ds > 0
\]
for some period $T > 0$ and all starting points $r_0 \geq R_0$ for some threshold $R_0$, assuming $\mathcal{R}(s)$ averages to zero or is dominated by the $I'-D'$ term.
\end{theorem}

\begin{proof}[Secular Growth of Knowledge Under Integrated Conditions]
\label{proof:bk3_knowledge_growth_integrated_condition}
If the integral condition in Thm.~\ref{thm:bk3_conditions_sustained_symbolic_growth} holds, then neglecting or assuming the average contribution of higher-order terms $\mathcal{R}(s)$ is small over the period $T$, the change in knowledge structure $\Delta K = K(r_0+T) - K(r_0)$ is positive. If this holds recurrently for all $r_0$ above some threshold $R_0$, it implies a secular growth trend in $K(r)$, even if there are local decreases within a period. If the condition fails, i.e., the integral is non-positive for sufficiently large $r_0$, then differentiation dominates or balances integration on average, leading to fragmentation, stagnation, or loss of symbolic coherence rather than sustained growth.
\end{proof}

\subsection{Conceptual Bridges and Symbolic Networks}

% Definition 3.2.9
\begin{definition}[Compressed Relational Structure] \label{definition:bk3__begindefinitioncompressed_relational_st}
A compressed relational structure $\sigma$ within a symbolic membrane $\mathcal{M}$ (Def.~\ref{definition:bk3__begindefinitionsymbolic_membrane}) is a lower-dimensional representation that preserves essential topological and dynamical features of a region $\omega \subset \mathcal{M}$:
\[
\sigma = \mathcal{C}(\omega)
\]
where $\mathcal{C}: 2^{\mathcal{M}} \rightarrow \Sigma$ is a compression operator mapping regions (subsets of $\mathcal{M}$) to a space of compressed structures $\Sigma$.
\end{definition}

% Definition 3.2.10
\begin{definition}[Symbolic Network] \label{definition:bk3__begindefinitionsymbolic_network}
A symbolic network $\mathcal{N}$ is a graph structure where:
\begin{enumerate}
    \item Nodes represent compressed relational structures $\{\sigma_i\}$ (Def.~\ref{definition:bk3__begindefinitioncompressed_relational_st}).
    \item Edges represent conceptual bridges (Definition~\ref{definition:bk3__begindefinitionconceptual_bridge}) between these structures.
    \item The network possesses a global stability functional $\mathcal{S}: \mathcal{N} \rightarrow \mathbb{R}_+$ measuring its overall coherence.
\end{enumerate}
\end{definition}

% Theorem 3.2.11
\begin{theorem}[Emergence of Symbolic Networks] \label{theorem:bk3__begintheorememergence_of_symbolic_netwo}
Under sustained symbolic growth conditions (Theorem~\ref{thm:bk3_conditions_sustained_symbolic_growth}), coupled symbolic membranes naturally generate symbolic networks (Def.~\ref{definition:bk3__begindefinitionsymbolic_network}) through the formation of compressed relational structures (Def.~\ref{definition:bk3__begindefinitioncompressed_relational_st}) and conceptual bridges (Def.~\ref{definition:bk3__begindefinitionconceptual_bridge}) (see proof~\ref{proof:bk3_sketch_symbolic_network_emergence}).
\end{theorem}

\begin{proof}[Sketch-Symbolic Network Emergence]
\label{proof:bk3_sketch_symbolic_network_emergence}
Sustained symbolic growth (Theorem~\ref{thm:bk3_conditions_sustained_symbolic_growth}) implies increasing internal organization and coherence $\kappa(x,r)$ within membranes (Definition~\ref{definition:bk3__begindefinitionsymbolic_knowledge_struc}). Regions $\omega \subset \mathcal{M}$ with high coherence become stable, identifiable structures. These coherent regions are candidates for compression via $\mathcal{C}$ into compressed relational structures $\sigma_i$ (Def.~\ref{definition:bk3__begindefinitioncompressed_relational_st}). Reflexive encodings $E_{ij}: \mathcal{M}_i \rightarrow \mathcal{M}_j$ (Definition~\ref{definition:bk3__begindefinitionreflexive_encoding}), particularly between highly coherent regions, establish conceptual bridges (Lemma~\ref{lem:bk3_reflexive_encodings_generate_conceptual_bridges}). These bridges act as edges connecting the compressed structures $\sigma_i$ (nodes) derived from different membranes (or even within the same membrane), forming a network $\mathcal{N}$ (Def.~\ref{definition:bk3__begindefinitionsymbolic_network}). The stability $\mathcal{S}$ of this network arises from the internal coherence of each $\sigma_i$ (related to $\kappa$ within $\omega_i$) and the fidelity (low distortion, structure preservation) of the conceptual bridges connecting them. (This supports Thm.~\ref{theorem:bk3__begintheorememergence_of_symbolic_netwo}).
\end{proof}

% Definition 3.2.12
\begin{definition}[Conceptual Bridge Sequence] \label{definition:bk3__begindefinitionconceptual_bridge_sequen}
The conceptual bridge sequence represents the progressive transformation and abstraction of symbolic structures:
\[
\Sigma_{\mathcal{M} \rightarrow \sigma}, \Sigma_{\sigma \rightarrow \Sigma}, \Sigma_{\Sigma \rightarrow \mathcal{N}}, \Sigma_{\mathcal{N} \rightarrow \mathcal{M}_{\text{meta}}}
\]
where each $\Sigma_{X \rightarrow Y}$ represents a conceptual bridge (Def.~\ref{definition:bk3__begindefinitionconceptual_bridge}) mapping structures of type $X$ to structures of type $Y$. This sequence maps membrane regions ($\mathcal{M}$, Def.~\ref{definition:bk3__begindefinitionsymbolic_membrane}) to compressed structures ($\sigma$, Def.~\ref{definition:bk3__begindefinitioncompressed_relational_st}), relates compressed structures to the space of such structures ($\Sigma$), organizes these into networks ($\mathcal{N}$, Def.~\ref{definition:bk3__begindefinitionsymbolic_network}), and potentially leads to the emergence of an encompassing meta-level symbolic membrane ($\mathcal{M}_{\text{meta}}$).
\end{definition}

% Theorem 3.2.13
\begin{theorem}[Closure of Conceptual Bridge Sequence] \label{thm:bk3_closure_conceptual_bridge_sequence}
The conceptual bridge sequence (Def.~\ref{definition:bk3__begindefinitionconceptual_bridge_sequen}) can form a closed loop, where the final meta-level membrane $\mathcal{M}_{\text{meta}}$ can itself host symbolic processes that influence the original membranes $\{\mathcal{M}_i\}$ (Def.~\ref{definition:bk3__begindefinitionsymbolic_membrane}).
\end{theorem}

\begin{proof}[Sketch-Evolutionary Dynamics]
\label{proof:bk3_sketch_evolutionary_dynamics}
The sequence (Def.~\ref{definition:bk3__begindefinitionconceptual_bridge_sequen}) progresses from membranes $\{\mathcal{M}_i\}$ to compressed structures $\{\sigma_k\}$, to networks $\mathcal{N}$ built upon these structures and their relations, and potentially to a meta-level representation $\mathcal{M}_{\text{meta}}$ that encodes the state or dynamics of the network $\mathcal{N}$. If this $\mathcal{M}_{\text{meta}}$ exists within the same overarching symbolic manifold $M$ (cf. Def.~\ref{definition:bk2_symbolic_probability_spa}) (or a related manifold that can interact with $M$), its internal dynamics (governed by its own drift field $D_{\text{meta}}$, etc.) can influence the environment or parameters affecting the original membranes $\{\mathcal{M}_i\}$. For instance, the state of $\mathcal{M}_{\text{meta}}$ could modulate the coupling strengths $\lambda_{ij}$ (Definition~\ref{definition:bk3__begindefinitioninduced_coupling_energy}), the response parameters $\eta_i$ (Theorem~\ref{theorem:bk3__begintheoremcoupling_induced_drift_modi}), or even the definitions of the reflexive encodings $E_{ij}$ (Definition~\ref{definition:bk3__begindefinitionreflexive_encoding}) between the lower-level membranes. This creates a feedback loop where the emergent higher-level structure ($\mathcal{M}_{\text{meta}}$ representing the network) regulates the behavior of the components ($\{\mathcal{M}_i\}$) from which it emerged. This closure enables self-modification, adaptation, and potentially more complex evolutionary dynamics within the symbolic system (supporting Thm.~\ref{thm:bk3_closure_conceptual_bridge_sequence}).
\end{proof}

\section{Symbolic Metabolism and Persistent Life} \label{sec:bk3_symbolic_metabolism_persistent_life}
\subsection{Symbolic Metabolism}

% Definition 3.3.1
\begin{definition}[Symbolic Metabolism] \label{definition:bk3__begindefinitionsymbolic_metabolism}
Symbolic metabolism refers to the regulated transformation and flow of symbolic structures across membranes (Def.~\ref{definition:bk3__begindefinitionsymbolic_membrane}) and conceptual bridges (Def.~\ref{definition:bk3__begindefinitionconceptual_bridge}) within a system, characterized by:
\begin{enumerate}
    \item Energy utilization: Transformation of symbolic potential energy (e.g., related to $H_{ij}$, Def.~\ref{definition:bk3__begindefinitioninduced_coupling_energy}) into structured information (e.g., maintaining $\rho_{ij}$, stable $\sigma_i$).
    \item Homeostasis: Maintenance of essential symbolic parameters (e.g., stability $S_i$, mutual information $I_{ij}$ from Def.~\ref{definition:bk3_symbolic_symbiosis}) within viable ranges despite perturbations.
    \item Adaptive response: Modification of internal processes (e.g., drift fields $D_i$, coupling $\Phi_{ij}$ from Def.~\ref{definition:bk3__begindefinitioncoupling_map}) in response to external or internal symbolic perturbations.
\end{enumerate}
\end{definition}

% Definition 3.3.2
\begin{definition}[Symbolic Metabolic Rate] \label{definition:bk3__begindefinitionsymbolic_metabolic_rate}
The symbolic metabolic rate $R_{\text{meta}}$ of a system of coupled symbolic membranes $\{\mathcal{M}_i\}$ (Def.~\ref{definition:bk3__begindefinitionsymbolic_membrane}) is defined as:
\[
R_{\text{meta}} = \sum_{i,j} \int_{\mathcal{M}_i \times \mathcal{M}_j} \rho_{ij}(x,y) \|\nabla_g H_{ij}(x,y)\|_g \, d\mu_g(x) \, d\mu_g(y)
\]
where:
\begin{itemize}
    \item $\rho_{ij}$ is the joint symbolic probability density (Def.~\ref{definition:bk2__symbolic_probability_density}) over the coupled membranes $\mathcal{M}_i$ and $\mathcal{M}_j$,
    \item $H_{ij}$ is the coupling Hamiltonian (energy) between membranes (Definition~\ref{definition:bk3__begindefinitioninduced_coupling_energy}),
    \item $\nabla_g$ is the gradient with respect to the symbolic metric $g$ (from Def.~\ref{definition:bk2_symbolic_probability_spa}) (acting on both $x$ and $y$ components, norm taken in the product tangent space),
    \item and the integral quantifies the total symbolic flux or activity driven by coupling-induced forces, weighted by the probability density.
\end{itemize}
\end{definition}

% Remark 3.3.3
\begin{remark} \label{remark:bk3__beginremark}
The symbolic metabolic rate $R_{\text{meta}}$ (Def.~\ref{definition:bk3__begindefinitionsymbolic_metabolic_rate}) measures the system's internal symbolic "activity" — the intensity of regulated information and energy flows that sustain structural coherence and dynamics across the coupled membranes. It reflects the magnitude of the forces mediating the interactions.
\end{remark}

% Definition 3.3.4
\begin{definition}[Autophagic Drift] \label{definition:bk3_autophagic_drift}
Autophagic drift is a symbolic phase in which agency $\mathcal{A}$ is suspended (cf.~\ref{corollary:bk9_emergence_of_moral_agency}) 
and symbolic drift $\mathcal{D}$ proceeds without immediate constraint (cf.~\ref{definition:bk1_proto_drift_field}). 
This phase allows symbolic membranes (cf.~\ref{definition:bk3__begindefinitionsymbolic_membrane}) 
to perform selective self-digestion, pruning unstable or incoherent forms and redistributing symbolic free energy (cf.~\ref{definition:bk5_symbolic_free_energy_und}).

It is metabolically essential: a regenerative drift cycle that supports long-term coherence (cf.~\ref{definition:bk3__begindefinitionsymbolic_metabolism}) 
by enabling spontaneous symbolic recomposition beneath the horizon of active regulation (cf.~\ref{definition:bk1_observer_relative_interpretability}).
\end{definition}

\subsection{Metabolic Stability and Regulation}

% Definition 3.3.4
\begin{definition}[Symbolic Homeostasis] \label{definition:bk3__begindefinitionsymbolic_homeostasis}
A symbolic system maintains homeostasis if, for a bounded range of perturbations $\delta$ (affecting, e.g., drift fields or external potentials), the symbolic metabolic rate $R_{\text{meta}}$ (Def.~\ref{definition:bk3__begindefinitionsymbolic_metabolic_rate}) remains within a stable operating band:
\[
R_{\text{min}} \leq R_{\text{meta}}(\delta) \leq R_{\text{max}}
\]
where $R_{\text{min}}, R_{\text{max}}$ are threshold bounds specific to the system's structure (e.g. membranes, Def.~\ref{definition:bk3__begindefinitionsymbolic_membrane}) and viability requirements.
\end{definition}

% Theorem 3.3.5
\begin{theorem}[Homeostatic Reflexes] \label{theorem:bk3__begintheoremhomeostatic_reflexes}
A symbolic system exhibits homeostatic reflexes if perturbations $\delta$ trigger compensatory adjustments $\Delta D_i$ in the drift fields (or other regulatory parameters like $\eta_i, \lambda_{ij}$ from Thm.~\ref{theorem:bk3__begintheoremcoupling_induced_drift_modi} and Def.~\ref{definition:bk3__begindefinitioninduced_coupling_energy}) such that the sensitivity of the metabolic rate (Def.~\ref{definition:bk3__begindefinitionsymbolic_metabolic_rate}) to the perturbation is bounded:
\[
\left| \frac{d R_{\text{meta}}}{d \delta} \right| \leq C
\]
for some bounded constant $C > 0$, across a specified operating regime. This implies that the system actively counteracts disturbances to maintain its metabolic rate (supporting Def.~\ref{definition:bk3__begindefinitionsymbolic_homeostasis}).
\end{theorem}

\begin{proof}[Sketch-Field Perturbation]
\label{proof:bk3_sketch_field_perturbation}
Perturbations $\delta$ might directly alter the drift fields $D_i$ or the coupling energies $H_{ij}$ (Def.~\ref{definition:bk3__begindefinitioninduced_coupling_energy}). Compensatory adjustments $\Delta D_i$ can arise from the feedback mechanisms inherent in the coupling (Theorem~\ref{theorem:bk3__begintheoremcoupling_induced_drift_modi}) or potentially from higher-level regulation (Theorem~\ref{thm:bk3_closure_conceptual_bridge_sequence}). If these adjustments counteract the effect of $\delta$ on the probability densities $\rho_{ij}$ and the coupling forces $\nabla_g H_{ij}$ sufficiently well, the overall change in $R_{\text{meta}}$ will be limited. The existence of such reflexes depends on the stability properties of the coupled system, particularly the effectiveness of the drift compensation mechanisms (Definition~\ref{definition:bk3_symbolic_symbiosis}, condition 3) and potentially active regulation loops. Bounded sensitivity implies robust homeostasis (Thm.~\ref{theorem:bk3__begintheoremhomeostatic_reflexes}).
\end{proof}

\subsection{Persistent Symbolic Life}

\begin{definition}[Symbolic Autopoiesis] \label{definition:bk3__begindefinitionsymbolic_autopoiesis}
A symbolic system exhibits autopoiesis (self-production and maintenance) if it sustains a closed loop of symbolic production, maintenance, and regulation of its own constituent components (membranes (Def.~\ref{definition:bk3__begindefinitionsymbolic_membrane}), coupling maps (Def.~\ref{definition:bk3__begindefinitioncoupling_map}), etc.), characterized by:
\begin{enumerate}
    \item Self-Maintenance: Membranes $\{\mathcal{M}_i\}$ persist over time due to internal stability (Theorem~\ref{theorem:bk3__begintheoremmembrane_stability_criteria}) and symbiotic stabilization (Definition~\ref{definition:bk3_symbolic_symbiosis}).
    \item Self-Modification: Reflexive encodings (Def.~\ref{definition:bk3__begindefinitionreflexive_encoding}) and coupling dynamics (e.g. Thm.~\ref{theorem:bk3__begintheoremcoupling_induced_drift_modi}, Def.~\ref{definition:bk3__begindefinitioninduced_coupling_energy}) allow the system to modify its own drift fields, coupling configurations, and potentially membrane boundaries or permeability in response to experience or internal states.
    \item Self-Extension: Conceptual bridges (Def.~\ref{definition:bk3__begindefinitionconceptual_bridge}) can evolve or be newly formed, allowing the system to incorporate new symbolic domains or refine its internal network structure ($\mathcal{N}$, Def.~\ref{definition:bk3__begindefinitionsymbolic_network}).
\end{enumerate}
\end{definition}

% Theorem 3.3.7
\begin{theorem}[Criteria for Persistent Symbolic Life] \label{thm:bk3_criteria_persistent_symbolic_life}
A symbolic system supports persistent symbolic life (understood as a dynamically stable, adaptive, and potentially growing symbolic organization) if:
\begin{enumerate}
    \item Symbolic metabolic rate $R_{\text{meta}}$ (Def.~\ref{definition:bk3__begindefinitionsymbolic_metabolic_rate}) remains within stable operating bands ($[R_{\text{min}}, R_{\text{max}}]$) indefinitely, indicating sustained regulated activity (Symbolic Homeostasis, Definition~\ref{definition:bk3__begindefinitionsymbolic_homeostasis}).
    \item Symbolic knowledge structure $K(r)$ (Def.~\ref{definition:bk3__begindefinitionsymbolic_knowledge_struc}) continues to grow recurrently (satisfying conditions like Theorem~\ref{thm:bk3_conditions_sustained_symbolic_growth}), indicating ongoing refinement and complexification.
    \item Symbiotic curvature $\kappa_{\text{symb}}$ (Def.~\ref{definition:bk3__begindefinitionsymbiotic_curvature}) remains strictly positive and bounded away from zero over time, ensuring persistent coupling, stability enhancement, and information exchange (Definition~\ref{definition:bk3_symbolic_symbiosis}, Theorem~\ref{theorem:bk3__begintheoremproperties_of_symbiotic_cur}).
\end{enumerate}
\end{theorem}

\begin{proof}[Sketch-Necessity for Continuous Operation]
\label{proof:bk3_sketch_necessity_for_continuous_operation}
Condition 1 (Homeostatic $R_{\text{meta}}$, Def.~\ref{definition:bk3__begindefinitionsymbolic_homeostasis}) ensures the system maintains the necessary internal symbolic activity and energy flow for continued operation without collapse or runaway processes. Condition 2 (Recurrent $K(r)$ growth, Def.~\ref{definition:bk3__begindefinitionsymbolic_knowledge_struc} and Thm.~\ref{thm:bk3_conditions_sustained_symbolic_growth}) ensures the system avoids stagnation or decay into trivial states, continuously refining its internal structure and potentially adapting. Condition 3 (Sustained $\kappa_{\text{symb}} > \epsilon > 0$, Def.~\ref{definition:bk3__begindefinitionsymbiotic_curvature} and Thm.~\ref{theorem:bk3__begintheoremproperties_of_symbiotic_cur}) ensures that the membranes remain effectively coupled and mutually supportive, preventing the system from dissociating into isolated, non-interacting components. Together, these conditions describe a system that actively maintains its organization, adapts its structure, and sustains the interactions necessary for a persistent, complex symbolic existence, analogous to biological life's persistence through metabolism, growth, and regulation (supporting Thm.~\ref{thm:bk3_criteria_persistent_symbolic_life}).
\end{proof}

\subsection{Toward Symbolic Evolution}

% Remark 3.3.8
\begin{remark} \label{remark:bk3__beginremark_1}
The emergence of persistent symbolic life (Theorem~\ref{thm:bk3_criteria_persistent_symbolic_life}), characterized by self-maintaining, self-modifying symbolic systems (Definition~\ref{definition:bk3__begindefinitionsymbolic_autopoiesis}), naturally leads to the conditions necessary for symbolic evolution. If we consider populations of such symbolic systems (or interacting membranes within a larger system), variations can arise through perturbations to drift fields (mutations) or changes in coupling. Differential stability and persistence (related to $S_i$, $\kappa_{\text{symb}}$, $K(r)$) provide a basis for selection, where more resilient or adaptive symbolic configurations are more likely to persist and influence future states. Coupling dynamics mediate interactions and competition/cooperation. Thus, the framework of symbolic thermodynamics and symbiosis potentially gives rise not merely to individual symbolic agents, but to entire ecosystems of evolving symbolic structures.
\end{remark}
% End of Book III content