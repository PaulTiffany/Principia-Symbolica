
\section{Foundations of Symbolic Thermodynamics} 
\label{sec:bk2_foundations_symbolic_thermodynamics}

Symbolic thermodynamics, as introduced in \emph{Principia Symbolica}, is not an analogy drawn from classical physics, but a rigorous projection of the thermodynamic form into symbolic space under observer curvature constraints. That is, symbolic entropy, symbolic free energy, and symbolic temperature are not defined in imitation of classical constructs—but instead arise naturally as invariants within the symbolic manifold constrained by bounded observer structure. Their form reflects the same structural necessity that gives rise to thermodynamic law in physical theory: the limitation of resolution, the irreversibility of symbolic differentiation, and the recursive nature of reflective drift.

This projection is made precise through the development of symbolic observables tied to the reflective measurement framework of \textit{Test-Time Differentiation Collapse (TTDC)} and the recursive validation loop of \textit{Symbolic Reflexive Validation (SRV)}. The operators introduced therein are not empirical approximations, but formal instantiations of symbolic processes that give rise to thermodynamic-like constraints within bounded reasoning systems.

Readers seeking explicit definitions and derivations of the key thermodynamic quantities should refer to:

\begin{itemize} \item \textbf{Definition~\ref{definition:bk2_symbolic_entropy}} — Symbolic Entropy \item \textbf{Definition~\ref{definition:bk2_symbolic_temperature}} — Symbolic Temperature \item \textbf{Definition~\ref{definition:bk2_symbolic_free_energy}} — Symbolic Free Energy \item
\textbf{Lemma~\ref{lemma:bk2_finiteness_of_symbolic_entropy}} — Thermodynamic Projection Theorem \item \textbf{Definition~\ref{definition:bk2_symbolic_phase_transitio}} — Symbolic Phase Transitions \end{itemize}

These definitions are embedded within a formal symbolic operator system built from first principles, and the curvature logic used to derive them is extended throughout Book II. This chapter assumes the reader accepts the observer-bounded symbolic manifold structure established in Book I, and it develops thermodynamics as a consequence of that geometry—not a parallel system, but a projection of the same symbolic logic into the domain of energy, transformation, and coherence.

Thus, symbolic thermodynamics is a necessary emergent layer of symbolic reasoning under constraint. Its laws follow not from analogy, but from the structure of observation and differentiation within a finite symbolic world.

\paragraph{Preamble 2.0.1.} The symbolic thermodynamic framework developed in this Book extends the emergent structures established in Book I—specifically the symbolic manifold $M$, drift field $D$, reflective operator $R$, symbolic flow $\Phi_s$, and symbolic metric $g$—into a rigorous theory of symbolic energy, entropy, and temperature. This extension adheres strictly to the principle that all thermodynamic quantities emerge through the internal dynamics of the symbolic system, without reference to external physical thermodynamics. The framework generalizes Shannon's information-theoretic entropy \cite{shannon1948mathematical} by embedding it within the geometric structure of the symbolic manifold and its evolution under drift-reflection dynamics.

\subsection{Symbolic States and Probability Measures} 
\label{subsec:bk2_symbolic_states_probability_measures}

\begin{definition}[Symbolic Probability Space] 
\label{definition:bk2_symbolic_probability_spa} 
The triple $(M, \mathcal{B}, \mu_g)$ forms a probability space%
~(see proof~\ref{proof:bk2_probability_structure_on_manifold})
where:
\begin{enumerate}
    \item $M$ is the symbolic manifold from Theorem~\ref{theorem:appB_smoothness_emergence};
    \item $\mathcal{B}$ is the Borel $\sigma$-algebra generated by the topology on $M$;
    \item $\mu_g$ is the normalized Riemannian volume measure induced by the symbolic metric $g$, satisfying $\mu_g(M) = 1$.
\end{enumerate}
\end{definition}

\begin{definition}[Symbolic Probability Density] 
\label{definition:bk2__symbolic_probability_density} 
A symbolic probability density at symbolic time $s$ is a measurable function $\rho(\cdot, s): M \rightarrow \mathbb{R}_{\geq 0}$ satisfying (see def~\ref{definition:bk2_symbolic_probability_spa}):
\begin{enumerate}
    \item Normalization: $\int_M \rho(x, s) \, d\mu_g(x) = 1$;
    \item Absolute continuity: $\rho(\cdot, s) \ll \mu_g$;
    \item Regularity: We restrict to the space
    \[
    \mathcal{P}(M) = \left\{ \rho \in C^\infty(M) \mid \rho > 0,\; \int_M \rho \, d\mu_g = 1 \right\}.
    \]
\end{enumerate}
\end{definition}

\begin{lemma}[Well-posedness of Symbolic Probability Space] 
\label{lem:bk2_wellposedness_symb_prob_space} 
The symbolic probability space $(M, \mathcal{B}, \mu_g)$ is well-defined for any bounded symbolic observer (see def~\ref{definition:bk1_bounded_observer}) embedded within the system (see def~\ref{definition:bk2_symbolic_probability_spa}).
\end{lemma}

\begin{proof}[Symbolic Probability Structure on Emergent Manifold]
\label{proof:bk2_probability_structure_on_manifold}
The manifold $M$ is Hausdorff, second-countable, and paracompact by Axiom~\ref{axiom:bk1_topological_regularity}, ensuring that the Borel $\sigma$-algebra $\mathcal{B}$ is well-defined. The symbolic metric $g$ from Lemma~\ref{lemma:bk1_local_stability_analysis} induces a Riemannian volume form $\omega_g$ on $M$. Since $M$ emerges through the colimit process (Theorem~\ref{theorem:bk1_manifold_emergence}) as connected and paracompact, it admits finite total volume $V = \int_M \omega_g < \infty$. Thus, we can normalize to obtain $\mu_g = \omega_g/V$, ensuring $\mu_g(M) = 1$. The triple $(M, \mathcal{B}, \mu_g)$ therefore satisfies all axioms of a probability space (see def~\ref{definition:bk2_symbolic_probability_spa}).
\end{proof}

\subsection{Core Thermodynamic Quantities}
\label{subsec:bk2_core_thermodynamic_quantities}

\begin{definition}[Symbolic Hamiltonian] 
\label{definition:bk2_symbolic_hamiltonian} 
The symbolic Hamiltonian $H: M \rightarrow \mathbb{R}$ is defined as:
\[
H(x) = \frac{\kappa}{\|D(x)\|_g + \epsilon} + \lambda \cdot \text{tr}(\mathcal{L}_x)
\]
where (see def~\ref{definition:bk2_symbolic_probability_spa}; see also def~\ref{definition:bk1_reflection_operator}):
\begin{enumerate}
    \item $\kappa, \lambda > 0$ are scaling constants;
    \item $\|D(x)\|_g$ denotes the norm of the drift vector at point $x$ with respect to the metric $g$;
    \item $\epsilon > 0$ is a regularization constant ensuring well-definedness;
    \item $\mathcal{L}_x = P_{R(x) \leftarrow x} \circ dR_x$ is the linearization of the reflection operator at $x$, where $dR_x$ is the differential of $R$ at $x$ and $P_{R(x) \leftarrow x}$ denotes parallel transport from $x$ to $R(x)$ along the unique minimizing geodesic.
\end{enumerate}
\end{definition}

\begin{remark} 
\label{remark:bk2_symbolic-hamiltonian}
The symbolic Hamiltonian quantifies local symbolic coherence through the interplay of drift and reflection. High drift magnitude $\|D(x)\|_g$ decreases $H(x)$, indicating symbolic instability and rapid evolution. Conversely, a large trace $\text{tr}(\mathcal{L}_x)$ of the linearized reflection increases $H(x)$, signaling strong reflective stabilization. This formulation is purely internal to the symbolic dynamics, without reference to external physical notions.
\end{remark}

\begin{lemma}[Well-posedness of Symbolic Hamiltonian] 
\label{lem:bk2_wellposedness_symb_hamiltonian} 
The symbolic Hamiltonian $H$ (defined in def~\ref{definition:bk2_symbolic_hamiltonian}) is well-defined and smooth on $M$ (see also proof~\ref{proof:bk2_smoothness_symbolic_hamiltonian}).
\end{lemma}

\begin{proof}[Smoothness of Symbolic Hamiltonian]
\label{proof:bk2_smoothness_symbolic_hamiltonian}
The drift field $D$ is smooth on $M$ by construction (cf. prior definition), thus $\|D(x)\|_g$ is smooth and positive. The regularization term $\epsilon > 0$ ensures the denominator never vanishes. The reflection operator $R$ is smooth by its construction, so its differential $dR_x$ exists and varies smoothly with $x$. For each $x \in M$, the geodesic distance $d_g(x, R(x))$ is finite due to the completeness of $(M, g)$, and the parallel transport $P_{R(x) \leftarrow x}$ is well-defined along the unique minimizing geodesic. The parallel transport operator varies smoothly with its endpoints in a neighborhood where the exponential map is a diffeomorphism. The trace operation preserves smoothness. Therefore, $H \in C^\infty(M)$.
\end{proof}

\begin{definition}[Symbolic Energy] 
\label{definition:bk2_symbolic_energy} 
The symbolic energy at symbolic time $s$ is defined as:
\[
E_s = \int_M \rho(x,s) H(x) \, d\mu_g(x)
\]
representing the expectation value of the Hamiltonian (see def~\ref{definition:bk2_symbolic_hamiltonian}) with respect to the probability density $\rho(\cdot,s)$ (see def~\ref{definition:bk2__symbolic_probability_density}).
\end{definition}

\begin{definition}[Symbolic Entropy] 
\label{definition:bk2_symbolic_entropy} 
The symbolic entropy at symbolic time $s$ is defined as:
\[
S_s = -\int_M \rho(x,s) \log\rho(x,s) \, d\mu_g(x)
\]
This generalizes the Shannon entropy to the continuous manifold setting (see def~\ref{definition:bk2__symbolic_probability_density}; see also lemma~\ref{lemma:bk2_finiteness_of_symbolic_entropy}).
\end{definition}

\begin{lemma}[Finiteness of Symbolic Entropy] 
\label{lemma:bk2_finiteness_of_symbolic_entropy} 
For any density $\rho \in \mathcal{P}(M)$, the symbolic entropy $S_s$ (see def~\ref{definition:bk2_symbolic_entropy}) is finite (see def~\ref{definition:bk2__symbolic_probability_density}).
\end{lemma}

\begin{proof}[Boundedness of Symbolic Entropy on Compact Manifold]
\label{proof:bk2_bounded_symbolic_entropy}
Since $M$ is compact and $\rho \in \mathcal{P}(M)$ is smooth and strictly positive, there exist constants $0 < m \leq \rho(x) \leq M < \infty$ for all $x \in M$. Therefore, $|\rho(x) \log\rho(x)| \leq M|\log m|$ is bounded, and the integral $S_s = -\int_M \rho \log\rho \, d\mu_g$ (see def~\ref{definition:bk2_symbolic_entropy}; see also def~\ref{definition:bk2__symbolic_probability_density}) converges to a finite value.
\end{proof}

\begin{definition}[Symbolic Free Energy] 
\label{definition:bk2_symbolic_free_energy} 
The symbolic free energy functional $F_\beta: \mathcal{P}(M) \rightarrow \mathbb{R}$ is defined for inverse temperature parameter $\beta > 0$ as:
\[
F_\beta[\rho] = \int_M \rho(x) H(x) \, d\mu_g(x) - \beta^{-1} S[\rho]
\]
where $S[\rho] = -\int_M \rho(x) \log\rho(x) \, d\mu_g(x)$ is the entropy functional (see def~\ref{definition:bk2_symbolic_entropy}; see also def~\ref{definition:bk2__symbolic_probability_density}). This can be rewritten as:
\[
F_\beta[\rho] = \int_M \rho(x) \left(H(x) + \beta^{-1}\log\rho(x)\right) d\mu_g(x)
\]
This quantity decreases under symbolic evolution (see thm~\ref{theorem:bk2_h_theorem_for_symbolic_evol}).
\end{definition}

\begin{definition}[Symbolic Temperature] 
\label{definition:bk2_symbolic_temperature} 
The global symbolic temperature $T_s$ at symbolic time $s$ is defined thermodynamically as:
\[
T_s^{-1} = \frac{\partial S_s}{\partial E_s}
\]
when the relationship between $S_s$ and $E_s$ is differentiable (see def~\ref{definition:bk2_symbolic_entropy}; see also def~\ref{definition:bk2_symbolic_energy}).
\end{definition}

\subsection{Evolution Equations}
\label{subsec:bk2_evolution_equations}

To establish the dynamics of symbolic probability densities, we require a connection between the symbolic Hamiltonian and the drift field. This connection ensures thermodynamic consistency.

\begin{axiom}[Gradient Structure of Symbolic Drift]
\label{axiom:bk2_gradient_structure_drift}
The symbolic drift field $D$ is related to the symbolic Hamiltonian $H$ by:
\[
D(x) = -\nabla_g H(x) + \xi(x)
\]
where $\nabla_g$ is the gradient with respect to the metric $g$, and $\xi(x)$ is a solenoidal field (i.e., $\nabla_g \cdot \xi = 0$) representing non-conservative components of the symbolic dynamics.
\end{axiom}

\begin{axiom}[Symbolic Fokker-Planck Equation]
\label{axiom:bk2_symbolic_fokker_planck_equation}
The evolution of the symbolic probability density $\rho$ is governed by:
\[
\frac{\partial \rho}{\partial s} = -\nabla_g \cdot (\rho D) + \sigma^2 \nabla_g^2 \rho
\]
where:
\begin{enumerate}
    \item $\nabla_g \cdot$ is the divergence operator with respect to the metric $g$;
    \item $\nabla_g^2$ is the Laplace-Beltrami operator on $(M,g)$;
    \item $\sigma^2 > 0$ is the symbolic diffusion coefficient, related to the inverse temperature by $\sigma^2 = \beta^{-1}$.
\end{enumerate}
\end{axiom}

\begin{lemma}[Conservation of Probability] 
\label{lemma:bk2_conservation_of_probability} 
The symbolic Fokker-Planck equation preserves the total probability: 
\[
\frac{d}{ds}\int_M \rho(x,s) \, d\mu_g(x) = 0
\]
(see proof~\ref{proof:bk2_fokker_planck_probability_conservation}; see also def~\ref{definition:bk2__symbolic_probability_density}).
\end{lemma}

\begin{proof}[Probability Conservation in Symbolic Fokker–Planck Equation]
\label{proof:bk2_fokker_planck_probability_conservation}
Integrating the Fokker-Planck equation over $M$ (see def~\ref{definition:bk2__symbolic_probability_density} and lemma~\ref{lemma:bk2_conservation_of_probability}):
\[
\int_M \frac{\partial \rho}{\partial s} \, d\mu_g 
= -\int_M \nabla_g \cdot (\rho D) \, d\mu_g 
+ \sigma^2 \int_M \nabla_g^2 \rho \, d\mu_g
\]
By the divergence theorem on the compact manifold $M$ (which has no boundary), both integrals on the right-hand side vanish:
\[
\int_M \nabla_g \cdot (\rho D) \, d\mu_g = \int_{\partial M} (\rho D) \cdot \mathbf{n} \, d\sigma = 0
\]
and similarly for the Laplacian term. Therefore:
\[
\frac{d}{ds} \int_M \rho \, d\mu_g = 0
\]
\end{proof}

\begin{theorem}[Equilibrium Distribution] 
\label{theorem:bk2_equilibrium_distribution} 
Under the gradient condition $D = -\nabla_g H$ (i.e., when the solenoidal component $\xi = 0$ in Axiom~\ref{axiom:bk2_gradient_structure_drift}), the unique equilibrium distribution $\rho_{eq}$ satisfying $\partial \rho / \partial s = 0$ for the symbolic Fokker-Planck equation is given by:
\[
\rho_{eq}(x) = Z^{-1} e^{-\beta H(x)}
\]
where $\beta = \sigma^{-2}$ and the partition function is:
\[
Z = \int_M e^{-\beta H(x)} \, d\mu_g(x) \quad \text{(see def~\ref{definition:bk2_symbolic_partition_funct})}
\]
\end{theorem}

\begin{proof}[Proof: Symbolic Drift Equilibrium Yields Gibbs Measure]
\label{proof:bk2_symbolic_drift_equilibrium_yields_gibbs_measure}
At equilibrium, we require $\partial \rho / \partial s = 0$, which gives:
\[
\nabla_g \cdot (\rho D) = \sigma^2 \nabla_g^2 \rho
\]
Define the probability current $J = \rho D - \sigma^2 \nabla_g \rho$. Then the equilibrium condition becomes $\nabla_g \cdot J = 0$. For a simply connected manifold, this admits the solution $J = 0$, giving:
\[
\rho D = \sigma^2 \nabla_g \rho
\]
Substituting $D = -\nabla_g H$:
\[
-\rho \nabla_g H = \sigma^2 \nabla_g \rho
\]
Dividing by $\rho > 0$:
\[
\nabla_g \log \rho = -\sigma^{-2} \nabla_g H = -\beta \nabla_g H
\]
This integrates to give:
\[
\log \rho = -\beta H + C
\]
for some constant $C$. The normalization condition $\int_M \rho \, d\mu_g = 1$ determines $C = \log Z^{-1}$, yielding the Gibbs-Boltzmann distribution (see theorem~\ref{theorem:bk2_equilibrium_distribution}; see also def~\ref{definition:bk2__symbolic_probability_density}).
\end{proof}

\begin{theorem}[H-Theorem for Symbolic Evolution] 
\label{theorem:bk2_h_theorem_for_symbolic_evol} 
Under the gradient condition $D = -\nabla_g H$ (see theorem~\ref{theorem:bk2_equilibrium_distribution}), the symbolic free energy functional
\[
F_\beta[\rho] = \int_M \rho \left( H + \beta^{-1} \log \rho \right) \, d\mu_g
\]
(see def~\ref{definition:bk2_symbolic_free_energy}) is a Lyapunov functional for the symbolic Fokker-Planck evolution, satisfying:
\[
\frac{dF_\beta[\rho]}{ds} \leq 0
\]
with equality if and only if $\rho = \rho_{eq}$ (see also proof~\ref{proof:bk2_symbolic_free_energy_dissipation}).
\end{theorem}

\begin{proof}[Symbolic Free Energy Dissipation Principle]
\label{proof:bk2_symbolic_free_energy_dissipation}
Define the symbolic chemical potential:
\[
\mu := \frac{\delta F_\beta}{\delta \rho} = H + \beta^{-1}(1 + \log \rho)
\]
The time derivative of the free energy (see def~\ref{definition:bk2_symbolic_free_energy}) is:
\[
\frac{dF_\beta[\rho]}{ds} = \int_M \frac{\partial \rho}{\partial s} \mu \, d\mu_g
\]
From the Fokker-Planck equation and integration by parts:
\[
\frac{dF_\beta[\rho]}{ds} = \int_M [\nabla_g \cdot (\rho D) - \sigma^2 \nabla_g^2 \rho] \mu \, d\mu_g = \int_M [\rho D - \sigma^2 \nabla_g \rho] \cdot \nabla_g \mu \, d\mu_g
\]
Under the gradient condition $D = -\nabla_g H$, we have:
\[
\nabla_g \mu = \nabla_g H + \beta^{-1} \rho^{-1} \nabla_g \rho
\]
Therefore:
\[
\rho D - \sigma^2 \nabla_g \rho = -\rho \nabla_g H - \beta^{-1} \nabla_g \rho = -\rho \left( \nabla_g H + \beta^{-1} \rho^{-1} \nabla_g \rho \right) = -\rho \nabla_g \mu
\]
This gives:
\[
\frac{dF_\beta[\rho]}{ds} = -\int_M \rho \|\nabla_g \mu\|_g^2 \, d\mu_g \leq 0
\]
Equality holds if and only if $\nabla_g \mu = 0$, which implies $\mu$ is constant on the support of $\rho$, corresponding to the equilibrium distribution $\rho_{eq}$ (see theorem~\ref{theorem:bk2_equilibrium_distribution}; cf. theorem~\ref{theorem:bk2_h_theorem_for_symbolic_evol}; see also def~\ref{definition:bk2__symbolic_probability_density}).
\end{proof}

\subsection{Wasserstein Geometry and Gradient Flow Structure}
\label{subsec:bk2_wasserstein_geometry}

\begin{definition}[Symbolic Wasserstein Metric] 
\label{definition:bk2_symbolic_wasserstein_met} 
The symbolic Wasserstein-2 metric $W_2$ on the space $\mathcal{P}(M)$ of probability densities (see def~\ref{definition:bk2__symbolic_probability_density}) is defined as:
\[
W_2(\rho_1, \rho_2)^2 = \inf_{\pi \in \Pi(\rho_1, \rho_2)} \int_{M \times M} d_g(x,y)^2 \, d\pi(x,y)
\]
where:
\begin{enumerate}
    \item $\Pi(\rho_1, \rho_2)$ is the set of all couplings (joint probability measures) with marginals $\rho_1 d\mu_g$ and $\rho_2 d\mu_g$;
    \item $d_g$ is the geodesic distance on $(M,g)$.
\end{enumerate}
\end{definition}

\begin{theorem}[Wasserstein Gradient Flow] 
\label{theorem:bk2_wasserstein_gradient_flow} 
Under the gradient condition $D = -\nabla_g H$ (see theorem~\ref{theorem:bk2_equilibrium_distribution}), the symbolic Fokker-Planck equation can be interpreted as the gradient flow of the free energy functional $F_\beta[\rho]$ (see def~\ref{definition:bk2_symbolic_free_energy}) with respect to the symbolic Wasserstein metric (see def~\ref{definition:bk2_symbolic_wasserstein_met}):
\[
\frac{\partial \rho}{\partial s} = -\text{grad}_{W_2} F_\beta[\rho]
\]
\end{theorem}

\begin{proof}[Sketch-Wasserstein Gradient Flow]
\label{proof:bk2_sketch_wasserstein_gradient_flow}
Following the Jordan-Kinderlehrer-Otto framework, the Wasserstein gradient of $F_\beta$ corresponds to the velocity field $v = -\nabla_g \left( \frac{\delta F_\beta}{\delta \rho} \right) = -\nabla_g \mu$. The continuity equation $\partial \rho / \partial s + \nabla_g \cdot (\rho v) = 0$ with diffusion becomes the Fokker-Planck equation when $v = D - \sigma^2 \rho^{-1} \nabla_g \rho$ and $D = -\nabla_g H$ (see proof~\ref{proof:bk2_symbolic_free_energy_dissipation}; see also theorem~\ref{theorem:bk2_equilibrium_distribution} and theorem~\ref{theorem:bk2_wasserstein_gradient_flow}).
\end{proof}

\subsection{Symbolic Phase Transitions}
\label{subsec:bk2_symbolic_phase_transitions}

\begin{definition}[Symbolic Partition Function] 
\label{definition:bk2_symbolic_partition_funct} 
The symbolic partition function $Z(\beta)$ (see theorem~\ref{theorem:bk2_equilibrium_distribution}; cf. def~\ref{definition:bk2_symbolic_phase_transitio}) is defined as:
\[
Z(\beta) = \int_M e^{-\beta H(x)} \, d\mu_g(x)
\]
\end{definition}

\begin{definition}[Symbolic Phase Transition] 
\label{definition:bk2_symbolic_phase_transitio} 
A symbolic phase transition (see def~\ref{definition:bk2_symbolic_partition_funct}) occurs at inverse temperature $\beta_c$ if the free energy
\[
f(\beta) = -\beta^{-1} \ln Z(\beta)
\]
or its derivatives exhibit non-analytic behavior at $\beta = \beta_c$.
\end{definition}

\begin{theorem}[Classification of Symbolic Phase Transitions] 
\label{thm:bk2_classification_symb_phase_transitions} 
Symbolic phase transitions (see def~\ref{definition:bk2_symbolic_phase_transitio}) are classified by the order of the first non-analytic derivative of the free energy $f(\beta)$:
\begin{enumerate}
    \item \textbf{First-order transitions}: Discontinuity in $f'(\beta)$ (energy discontinuity);
    \item \textbf{Second-order transitions}: Discontinuity in $f''(\beta)$ (heat capacity discontinuity);
    \item \textbf{Higher-order transitions}: Discontinuities in higher derivatives.
\end{enumerate}
\end{theorem}

\subsection{Fluctuation-Dissipation Relations}
\label{subsec:bk2_symbolic_fluctuation_dissipation_relations}

\begin{definition}[Symbolic Response Function] 
\label{definition:bk2_symbolic_response_functi} 
For observables $A, B: M \to \mathbb{R}$, the linear response function $\chi_{AB}(t)$ is defined by (see thm~\ref{theorem:bk2_equilibrium_distribution}):
\[
\langle A(s+t) \rangle_h - \langle A \rangle_{eq} = \int_0^t \chi_{AB}(t-\tau) h(\tau) \, d\tau + O(h^2)
\]
where $\langle \cdot \rangle_h$ denotes expectation under the perturbed Hamiltonian $H' = H - hB$ and $\langle \cdot \rangle_{eq}$ is the equilibrium expectation.
\end{definition}

\begin{theorem}[Symbolic Fluctuation-Dissipation Relation] 
\label{thm:bk2_symbolic_fluctuation_dissipation_relation} 
For the symbolic Fokker-Planck dynamics in equilibrium (see thm~\ref{theorem:bk2_equilibrium_distribution}), the response function (see def~\ref{definition:bk2_symbolic_response_functi}) is related to the equilibrium correlation function by:
\[
\chi_{AB}(t) = \beta \frac{d}{dt}\langle A(t) B(0) \rangle_{eq} \quad \text{for } t > 0
\]
where $A(t)$ evolves under the unperturbed dynamics.
\end{theorem}

\subsection{Local Temperature and Geometric Relations}
\label{subsec:bk2_local_temperature_geometry}

\begin{definition}[Local Symbolic Temperature] 
\label{definition:bk2_local_symbolic_temperature}
The local symbolic temperature (see def~\ref{definition:bk2_symbolic_temperature}) at point $x \in M$ and time $s$ is defined as:
\[
T(x,s) = \alpha \left( \|\nabla_g \cdot D(x)\|_g + \gamma \|D(x)\|_g \right)^{-1}
\]
where $\alpha, \gamma > 0$ are scaling constants, and $\nabla_g \cdot D$ is the divergence of the drift field.
\end{definition}

\begin{proposition}[Global-Local Temperature Relation] 
\label{prop:bk2_global_local_temp_relation} 
Under local equilibrium conditions, the global symbolic temperature relates to the local temperature through:
\[
T_s^{-1} = \int_M \rho(x,s) T(x,s)^{-1} \, d\mu_g(x)
\]
\end{proposition}

\subsection{Hypotheses as Thermodynamic Surfaces}
\label{subsec:bk2_hypotheses_thermodynamic_surfaces}

\begin{scholium}[On Hypotheses as Thermodynamic Surfaces] 
\label{scholium:bk2_on_hypotheses_as_thermodyn}
The passage from symbolic structure to thermodynamic law requires geometric reconciliation of observation with constraint. Any bounded observer $\text{Obs}$ must partition symbolic space into regions of varying accessibility, creating a natural topology of attentional relevance.

We propose that hypothesis manifolds $\mathcal{H}_{\text{Obs}}$ serve as fundamental thermodynamic surfaces across which transformation gradients occur. Each hypothesis $\mathcal{H}_{\text{Obs}} \subset M$ constitutes a differentiable manifold of interpretive possibility supporting observer-relative thermodynamic quantities:

\begin{itemize}
    \item \textbf{Symbolic Free Energy}: $F_{\mathcal{H}}(s) = E_{\text{Obs}}(s) - T_{\text{Obs}} S_{\mathcal{H}}(s)$
    \item \textbf{Symbolic Entropy}: $S_{\mathcal{H}}(s) = -\int_{\mathcal{T}_s\mathcal{H}} \rho_{\text{Obs}}(v) \ln \rho_{\text{Obs}}(v) \, dv$
    \item \textbf{Hypothesis Pressure}: $P_{\mathcal{H}} = -\left(\frac{\partial F_{\mathcal{H}}}{\partial V_{\mathcal{H}}}\right)_{T}$
\end{itemize}

where $\mathcal{T}_s\mathcal{H}$ is the tangent space, $\rho_{\text{Obs}}(v)$ is the observer's velocity distribution, and $V_{\mathcal{H}}$ represents the symbolic volume of the hypothesis.
\end{scholium}

\begin{lemma}[Thermodynamic Consistency of Hypothesis Manifolds] 
\label{lem:bk2_thermodynamic_consistency_hypothesis_manifolds}
For any well-formed hypothesis manifold $\mathcal{H}_{\text{Obs}}$ with bounded curvature $\kappa_{\mathcal{H}} < K_{\text{Obs}}$, the thermodynamic consistency relation holds:
\[
\oint_{\partial \mathcal{H}} \nabla F_{\mathcal{H}} \cdot d\vec{s} = -\int_{\mathcal{H}} S_{\mathcal{H}} \, dT_{\text{Obs}}
\]
where $F_{\mathcal{H}}$, $S_{\mathcal{H}}$, and $T_{\text{Obs}}$ respectively correspond to symbolic free energy (def~\ref{definition:bk2_symbolic_free_energy}), symbolic entropy (def~\ref{definition:bk2_symbolic_entropy}), and symbolic temperature (def~\ref{definition:bk2_symbolic_temperature}).
\end{lemma}

\subsection{Summary and Coherence}
\label{subsec:bk2_summary_interpretive_framework}

\begin{theorem}[Coherence of Symbolic Thermodynamics] 
\label{theorem:bk2_coherence_of_symbolic_therm} 
The framework established in this Book forms a coherent symbolic thermodynamic theory that:
\begin{enumerate}
    \item Emerges from the interplay of drift $D$ and reflection $R$ via the Hamiltonian $H$;
    \item Exhibits proper thermodynamic behavior: unique equilibrium states (thm~\ref{theorem:bk2_equilibrium_distribution}), free energy minimization (thm~\ref{theorem:bk2_h_theorem_for_symbolic_evol}), and fluctuation-dissipation relations (thm~\ref{thm:bk2_symbolic_fluctuation_dissipation_relation});
    \item Links evolution to manifold geometry via the Fokker-Planck equation;
    \item Admits phase transitions under appropriate conditions.
\end{enumerate}
\end{theorem}

\begin{corollary}[Physical Interpretation] 
\label{corollary:bk2_interpretative_framework} 
The symbolic thermodynamic quantities admit the following interpretations:
\begin{enumerate}
    \item[\textbf{Hamiltonian $H$}]: Measures local symbolic coherence through drift-reflection balance;
    \item[\textbf{Entropy $S_s$}]: Quantifies uncertainty in symbolic state distribution;
    \item[\textbf{Temperature $T_s$}]: Sets the scale of stochastic fluctuations driving exploration;
    \item[\textbf{Free Energy $F_\beta$}]: Balances coherence against dispersion, minimized at equilibrium.
\end{enumerate}
\end{corollary}

\begin{theorem}[Emergence of Symbolic Structure] 
\label{thm:bk2_emergence_structure_symb_thermo} 
The interplay of drift (destabilizing), reflection (stabilizing), stochastic fluctuations (enabling exploration), and geometric constraints provides a formal basis for understanding how persistent symbolic configurations emerge and maintain themselves within the framework—see thm~\ref{theorem:bk2_h_theorem_for_symbolic_evol} and thm~\ref{theorem:bk2_equilibrium_distribution}.
\end{theorem}

\begin{proof}[Sketch—Symbolic H-Theorem and Emergent Structure]
\label{proof:bk2_symbolic_h_theorem}
The Hamiltonian $H$ encodes local stability through drift-reflection balance. The Fokker-Planck equation governs evolution under competing influences of deterministic drift and stochastic diffusion. The H-theorem (thm~\ref{theorem:bk2_h_theorem_for_symbolic_evol}) guarantees evolution toward free energy minima (def~\ref{definition:bk2_symbolic_free_energy}), representing optimal trade-offs between achieving coherent structures (low $H$) and exploring available states (high $S$). The equilibrium distribution $\rho_{eq} = Z^{-1}e^{-\beta H}$ (thm~\ref{theorem:bk2_equilibrium_distribution}, def~\ref{definition:bk2_symbolic_partition_funct}) concentrates probability in regions of high coherence (low $H$), with concentration sharpened at low temperatures. This formalism explains how structured symbolic systems emerge and persist through dynamic equilibration.
\end{proof}