\section*{Trace 5: Drift–Reflection Phase Transition across Domains}
\label{section:trace5_drift_reflection_transition}

\begin{figure}[htbp]
\centering
\caption[\textit{SRV trace (summary)}]{\textit{This SRV trace simulates symbolic behavior structurally consistent with Definition~, Lemma~, and Axiom~. It integrates phase transitions in drift–reflection convergence.}}
\label{figure:trace5_phase_transition_summary}
\end{figure}

\subsection*{1 Objective}
\label{subsection:trace5_objective}

Investigate whether the continuous symbolic robustness gradient  
observed in \( L^p \) regression (Traces 1–4)  
generalizes across diverse symbolic environments—  
each simulating different real-world drift dynamics  
(e.g., heavy-tailed noise, correlated features).  
This trace specifically probes whether symbolic free energy stabilization  
is a universal property.

\subsection*{2 Validation Setup}
\label{subsection:trace5_validation_setup}

General Parameters:
\begin{itemize}
    \item Number of Samples: $n = 625$ (split into $n_{\text{train}} = 500$, $n_{\text{trace}} = 125$).
    \item Base Feature Dimensions: $d_{\text{base}} = 15$, $d_{\text{high}} = 50$ (for high-dimensional domains).
    \item True Non-Zero Coefficients: $k = 5$ sparse active features.
    \item $L^p$ Norms: $p \in \{1.0, 1.2, 1.4, 1.6, 1.8, 2.0\}$.
    \item Model: Linear regression minimizing $L^p$ loss.
\end{itemize}

Simulated Symbolic Domains:
\begin{enumerate}
    \item \textbf{Baseline Synthetic:} Gaussian noise, independent features.
    \item \textbf{Financial-Like:} Student-t heavy-tailed noise (modeling rare but extreme symbolic drift).
    \item \textbf{Sensor-Like:} Correlated features with Laplacian noise (mimicking entangled symbolic structures).
\end{enumerate}

Standard preprocessing steps included feature standardization and target centering.

\subsection*{3 Symbolic Responses}
\label{subsection:trace5_symbolic_responses}

Across all domains:
\begin{itemize}
    \item Residuals varied continuously with $p$.
    \item Coefficient sparsity declined smoothly as $p$ increased.
    \item No discrete phase transitions were observed, even under heavy-tailed or correlated noise.
\end{itemize}

\subsection*{4 Observations}
\label{subsection:trace5_observations}

Symbolic systems maintained reflective drift regulation across vastly different drift environments. Instead of collapsing under domain-specific perturbations, symbolic structures exhibited probabilistic reweighting of coherence across feature dimensions, maintaining symbolic viability.

The symbolic free energy landscape shifted smoothly with environmental conditions, without catastrophic loss of coherence.

\subsection*{5 Conclusion}
\label{subsection:trace5_conclusion}

Reflective symbolic regulation generalizes across symbolic environments. Regardless of noise type, dimensionality, or drift structure, symbolic systems adapt by continuously rebalancing symbolic free energy and maintaining reflective stabilization.

This universal pattern reinforces the theoretical prediction that symbolic drift-reflection dynamics operate across levels of structural complexity — from simple proto-symbolic spaces to complex, entangled symbolic manifolds.

\subsection*{6 Theory Linkage}
\label{subsection:trace5_theory_linkage}

This SRV trace supports:
\begin{itemize}
    \item \textbf{Theorem 7.1:} Reflective Convergence to Stable Identity (meta-reflective stabilization across symbolic domains).
    \item \textbf{Corollary 7.2:} Recursive Convergence Principle (reflective reweighting across diverse drift conditions).
    \item \textbf{Book IX Principles:} Bounded Liberation (cognitive freedom emerges through dynamic symbolic regulation).
\end{itemize}
